\documentclass{article}

\usepackage[utf8]{inputenc}
\usepackage[T1]{fontenc}
\usepackage{geometry}
\usepackage{graphicx}   %ALLOWS INSERTING IMAGES (AND MAYBE SOME MORE STUFF) %
\usepackage{caption}
\usepackage{subcaption} %ALLOWS CAPTIONS FOR SUBFIGURES%
\usepackage{verbatim} % ALLOWS MULTILINE COMMENTS WITH \begin{comment} AND \end{comment} %
\usepackage{amsmath,amssymb} % PERMETTE L'USO DI SIMBOLI MATEMATICI "AVANZATI" COME 'MINORE O UGUALE' E ALTRI %
\usepackage{amstext}
\usepackage{enumitem} %PERMETTE L'USO DI ELENCHI NUMERATI
\usepackage[section]{placeins}

\geometry{a4paper}
                        %Una riga vuota tra due scritte spazia di una riga anche sul pdf. Andare a capo una volta non provoca nulla nel pdf%
\usepackage[italian]{babel}
%\usepackage[french, italian]{babel} non funzione per motivi a me ignoti
\frenchspacing 

%%%  INTESTAZIONE  %%%

\title{Relazione dell'esperimento di misura della velocità della luce}
\author{Lorenzo Ramella, Alessandro Matteo Rossi, Marco Tambini}
\date{\today}


%%%  INIZIO DOCUMENTO  %%%

\begin{document}
\maketitle

%%%  ABSTRACT  %%%

\begin{abstract}
L’esperimento si propone di misurare la velocità delle luce usando il metodo di Focault. %%da modificare
\end{abstract}

%%%  INDICE  %%%%

\tableofcontents
\newpage

%%%  INTRODUZIONE TEORICA  %%%

\section{Introduzione teorica}
Il metodo di Focault per la misura della velocità della luce consiste nell'uso di uno specchio rotante, che riflette la luce emessa da una sorgente su di uno specchio 
concavo. 

Una sorgente luminosa $S$ emette una luce che, opportunamente diaframmata da una lente $L_1$, attraversa una lastra semitrasparente angolata di 45° rispetto alla 
direzione del fascio. Una lente $L_2$ focalizza il fascio nel punto $S'$ sullo specchio concavo, dopo essere stata deflessa dallo specchio rotante. La luce riflessa 
dallo specchio concavo viene deflessa nuovamente dallo specchio rotante, che nel frattempo ha ruotato di un angolo 

\begin{equation}
\alpha = \omega \frac{2D}{c}
\end{equation}

dove $\omega$ è la velocità angolare dello specchio e $D$ è la distanza tra lo specchio rotante e lo specchio concavo.

Il fascio luminoso di ritorno sulla lente $L_2$ viene focalizzato come se provenisse da una sorgente $S''$ spostata da $S'$ di una quantità 

\begin{equation}
\Delta = 2 \alpha D
\label{DELTA}
\end{equation}

Tenendo presente che il fattore di amplificazione $G$ della lente $L_2$ è esprimibile mediante la seguente relazione:

\begin{equation}
G=\frac{b}{D+a}
\label{G}
\end{equation}

dove $b$ è la distanza tra $L_2$ e la sorgente luminosa e $a$  la distanza tra $L_2$ e lo specchio rotante. Lo spostamento laterale $\delta$ dell'immagine si ottiene quindi
combinando le equazioni (\ref{DELTA}) e (\ref{G})
\begin{equation}
\delta = G\Delta =\frac{2\alpha b D}{D + a}=\frac{4 D^2 b \omega}{c(D+a)}
\label{delta}
\end{equation}

È possibile ricavare $b$ dalla legge dei punti coniugati. Sapendo che:

\begin{equation}
\frac{1}{b}+\frac{1}{D+a}=\frac{1}{f_2}
\label{punticoniug}
\end{equation}

Si deduce che:

\begin{equation}
b=\frac{f_2(D+a)}{D+a-f_2}
\end{equation}

E quindi, conoscendo l'espressione di $\delta$ dalla equazione (\ref{delta}), in cui compare $c$, e avendo ricavato l'unica variabile incognita $b$ dall'equazione
(\ref{punticoniug}), si ottiene:

\begin{equation}
c = \frac{4f_2D^2(\omega-\omega_0)}{[(D+a+f_2)\Delta\delta]}
\label{c_equation}
\end{equation}

dove $\omega_0$ è la velocità angolare iniziale e $\Delta\delta = \omega - \omega_0$ è lo spostamento dell'immagine della sorgente luminosa.

\begin{figure}[h]
    \centering
        \includegraphics[width=0.6\linewidth]{IntroTeorica1.JPG}
    \caption{Schema apparato - intorno del beam-splitter}
\end{figure}

%%%  PROGETTAZIONE DELL'ESPERIMENTO  %%%

\newpage

\section{Progettazione dell'esperimento}

\begin{figure}[h] %here, top, bottom, ! = forza posizionamento dove latex riesce, H mettilo dove cazzo dico io%
    \centering
    \includegraphics[width=0.6\linewidth]{Progettazione1.JPG}
    \caption{Schema dell'apparato di laboratorio}
    \label{schema_apparato}
\end{figure}

Il banco ottico di laboratorio era composto da una sorgente luminosa e un sistema di lenti e specchi, il tutto ancorato saldamente alle varie parti del banco mediante 
morse.

Prima di cominciare le misure di $c$ è necessario assicurarsi che gli elementi del banco ottico siano correttamente posizionati.

\vspace{3mm}

Per fare ciò principio abbiamo misurato la lunghezza del cammino ottico $D$ del raggio luminoso dallo specchio rotante allo specchio concavo. Per fare ciò abbiamo raccolto 
4 misure per le distanze tra specchio rotante e specchio 1 $d_{rot,1}$, tra specchio 1 e specchio 2 $d_{1,2}$ e tra specchio 2 e specchio concavo $d_{2,conc}$, per poi 
ottenere tramite una somma delle loro medie aritmetiche (con incertezza $\sigma d = \sigma D = 0,01 m$) il valore di $D$. Gli strumenti utilizzati in questa fase sono
stati due rotelle metriche di portata $15m$ e $3m$.

Come mostrato in Figura \ref{schema_apparato}, la sorgente luminosa coerente è un laser A CHE COSA?? fissato magneticamente a un binario graduato. Per evitare danni agli
occhi abbiamo adoperato una coppia di lamine polaroid, con base magnetica, poste sul binario subito dopo il laser.
è necessario porre una prima lente $L_1$ $70 mm$, seguendo la scala graduata con la sensibilità del millimetro, per poter fare convergere la luce proveniente "dall'infinito"
\begin{comment}(le dimensioni lineari della lente sono molto ridotte rispetto alla distanza tra laser e $L_2$)\end{comment} alla sua distanza focale, che il produttore 
della lente dichiara essere a una distanza $f_1 = 0,048 m$. A una distanza di $180mm$ dall'inizio della scala graduata è posto il beam-splitter, un elemento dotato di uno
specchio semiriflettente orientabile rispetto alla luce incidente, che permette di trasmettere la luce in arrivo alla lente $L_2$, se angolato di $45^\circ$. La lente $L_2$, con 
focale dichiarata dal produttore di $f_2=0,252m$, è posta a distanza $b = ???$ dal beam-splitter e trasmette a sua volta la luce allo specchio rotante. 
Questo sistema di lenti serve per ottenere un'onda piana, e questo è possibile poichè la somma delle focali di $L_1$ ed $L_2$ è maggiore della loro distanza. 
Il beam-splitter è uno strumento dotato di uno specchio


Per permette alla luce di percorrere il sistema di specchi che va dallo specchio rotante, allo specchio $S'$, quindi allo specchio $S''$ e infine allo specchio concavo 
bisogna inizialmente accertarsi, posizionando una squadretta forata con base magnetica sul binario, che la luce incida sul centro dello specchio rotante.
Contemporaneamente bisogna agire sulla cinghia dello specchio rotante per permettere al raggio incidente di raggiungere la parte riflettente dello specchio e quindi 
subire una riflessione verso il centro (approssimativamente) di $S'$.
Fatto questo si regola l'inclinazione di $S'$ per riflettere il raggio proveniente dallo specchio rotante approssimativamente nel centro di $S''$, attraverso due
viti micrometriche poste dietro lo specchio. Si ripete quest'ultimo procedimento per portare il raggio dal centro di $S''$ al centro dello specchio concavo.

Specifichiamo che a causa della complessità della misura della focale di una lente le focali e le posizioni corrette per lenti e beam-splitter vengono fornite dai 
docenti, e le consideriamo affette da un'incertezza sistematica trascurabile.

\vspace{3mm}

Raggiunta la condizione di raggio che percorre tutta la lunghezza $D$, è necessario agire nuovamente sulle viti micrometriche degli specchi per fare tornare il raggio
luminoso indietro fino al beam-splitter ripercorrendo il suo cammino di andata. Può essere in questa fase utile utilizzare un foglio di carta millimetrata da posizionare
a mezz'aria dove si pensa che i raggi passino, per verificare spostandolo sempre più verso lo "specchio di destinazione" che i raggi coincidano e che il puntino luminoso
del raggio di andata non si discosti da quello del raggio di ritorno.

In ultimo si è provveduto a regolare la messa a fuoco del microscopio, dopo averlo inserito nel foro sulla parte superiore del beam-splitter, fissandolo a una data altezza
all'interno del foro per mezzo di una vite. Il microscopio è utile per osservare il raggio di "ritorno" del laser, poichè in questo secondo passaggio per lo specchio, 
angolato di $45^\circ$, il raggio urta la parte riflettente dello specchio e viene deviato verso il cannocchiale del beam-splitter, permettendone l'osservazione. è 
possibile leggere la posizione di questo punto riflesso agendo su una vite micrometrica posta sul beam-splitter, che permette di centrare il puntino luminoso con un 
crocifilo, garantendo una precisione di $\sigma = 10 \mu m$.

\vspace{3mm}

\underline{Problemi tecnici riscontrati:} 

Sebbene sia abbastanza agevole misurare le distanze tra specchi, posizionare le lenti e il beam-splitter e fondamentale già da questa prima fase prestare la 
/underline{massima attenzione} a non toccare il banco di lavoro (singolare, sebbene $S'$ e $S''$ fossero posti su un secondo tavolo, il che rendeva utile misurare alcune
distanze con la rotella di portata $15m$). 
Questo potrebbe compromettere gravemente la taratura del banco.

È invece complicato e dispendioso a livello di tempo correggere la posizione degli specchi. Infatti mentre uno dei due sperimentatori lavorava sulle viti micrometriche,
l'altro aveva il compito di tracciare il percorso del raggio o osservandone la posizione sulle pareti (se e quando possibile) o aiutandosi con la carta millimetrata 
quando esso era prossimo a raggiungere il centro dello specchio.

Il compito certamente più arduo è fare ritornare il raggio di ritorno su quello di andata, poichè la carta millimetrata è l'unico strumento valido. Durante il suo utilizzo
il rischio di coprire il raggio di andata (facendo scomparire quello di ritorno) nel tentativo di vederli entrambi sul bordo del foglio è alto.

%%%  MISURE  %%%

\section{Misure}

\subsection{Raccolta dati}
L'esperienza è stata svolta da Rossi in collaborazione con Tambini e Ramella in collaborazione con Redaelli. Si è deciso di analizzare i due set di misure, fino al calcolo 
della miglior stima della velocità della luce $c_{best}$, separatamente. Questo a causa di differenze:

\begin{enumerate}
    \item Sulla misura di $D$ poichè sono stati usati due apparati dello stesso modello ma fisicamente differenti, e quindi si sono presentate differenze.
    \item Sul motore dello specchio rotante utilizzato. Il gruppo Rossi-Tambini ha utilizzato un motore per lo specchio rotante che permetteva solo la regolazione in 
            senso orario e antiorario con una frequenza di rotazione di $\nu = 1500 Hz$ e $\nu = 750 Hz$. Mentre il motore del gruppo Ramella-Redaelli permetteva una rotazione
            oraria e antioraria con frequenza variabile fino a circa i $1500 Hz$.
\end{enumerate}



\subsection{Analisi dati - Misure Rossi e Tambini}

L'apparato utilizzato da Rossi e Tambini era provvisto di motorino con frequenza di rotazione non regolabile con continuità, ma solo a $\nu=750Hz$ e $\nu=1500Hz$
in senso orario e antiorario. Dopo avere raccolto i dati relativi alla lunghezza del cammino ottico abbiamo adoperato il foglio di calcolo per ricavarne un valore pari 
a $D = 13,60 \pm 0,01 m$ (Sezione \ref{RT} - Figura \ref{RT_D}).

Si specifica che durante la trattazione di questo insieme di misure si indicheranno nel testo e nelle tabelle le rotazioni in senso orario con segno positivo, mentre 
quelle antiorarie con segno negativo. Questo non inficia i calcoli di quantità ricavate in maniera indiretta poichè le frequenze di rotazione sono state trattate in
modulo.

Siccome l'equazione \ref{c_equation} è risolubile inserendo dei valori di partenza $\omega_0$ e $\delta_0$ e dei valori finali $\omega$ e $\delta$, abbiamo proceduto
raccogliendo coppie di misure, rispettivamente per valori iniziali e finali, per poi calcolare da ogni coppia un valore della velocità della luce $c$. I dati raccolti
sono riportati in Sezione \ref{RT} - Figura \ref{RT_DatiRaccolti}.
Il valore $\delta_0$ è stato misurato mettendo lo specchio in rotazione a una certa $\nu$, dopo avere aspettato che l'immagine del laser riflessa visibile nel microscopio
del beam-splitter si fermasse in un punto. Infatti con l'utilizzo della vite micrometrica abbiamo centrato il puntino luminoso e ne abbiamo letto la posizione.
Analogamente è stato fatto per il valore $\delta$: abbiamo impostato il motorino su una diversa velocità, osservato lo spostamento del puntino e ne abbiamo misurato la
posizione.

Segnaliamo che durante l'esperienza in laboratorio abbiamo accidentalmente urtato per due volte il beam-splitter. è infatti possibile notare, osservando i dati raccolti 
(Sezione \ref{RT} - Figura \ref{RT_DatiRaccolti}), come i valori di $\delta_0$ e $\delta$ raccolti per uguali frequenze di rotazione presentino valori molto differenti 
tra loro durante le varie misure.

Abbiamo deciso di analizzare e quantificare gli effetti di questi urti. Abbiamo ipotizzato che non fosse possibile non avere alcun effetto provocato da queste compromissioni
dell'apparato di misura. Inoltre abbiamo supposto che la misura 38 della in \ref{RT} - Figura \ref{RT_DatiRaccolti} fosse stata pesantamente affetta da un errore causato 
da questi urti, poichè riportava un surreale valore della velocità della luce pari a $c_{38}\approx 4,788 \cdot10^8 \pm 2,0 \cdot10^6 m/s$, con un errore molto maggiore 
rispetto a tutte le altre misure, come si vedrà nei paragrafi successivi. 

Pertanto abbiamo suddiviso i valori $\delta_0$ e $\delta$ raccolti in base alla frequenza di rotazione a cui sono stati raccolti ($\nu=-750Hz$, $\nu=750Hz$, 
$\nu=-1500Hz$, $\nu=1500Hz$) e ne abbiamo fatto un grafico, come riportato in \ref{RT} - Figure (\ref{Tabs}), (\ref{Graf_750}), (\ref{Graf_1500}), (\ref{Graf_-750}), 
(\ref{Graf_-1500}). 
Già da un primo confronto qualitativo, osservando i grafici, era possibile osservare come le misure di $\delta_0$ e $\delta$ fossero divise in tre insiemi. Un'analisi 
quantitativa fatta applicando una distribuzione normale a ogni insieme per ogni $\nu$ (\ref{RT} - Figura (\ref{Gauss})) mostra come questi tre insiemi siano altamente incompatibili
(probabilità inferiore a $1\%$), con $z\gtrsim 3$. 
Ne è risultato che il confine tra questi gruppi di misure sono proprio quei dati raccolti prima e dopo l'urto contro il beam splitter, ovvero le misure 22 e 38.

Nonostante questi urti, le misure risultano essere coerenti con il valore vero della velocità della luce $c_{vero}\approx 2,998\cdot10^8 m/s$, eccezion fatta per la misura 38.
In questo specifico caso l'urto è avvenuto tra le due misure e infatti il valore stimato della velocità della luce per questa misura risulta essere $c_{38}=4,788\cdot10^8 \pm
2,0\cdot10^6 m/s$. Abbiamo deciso di procedere al rigetto di questa misura, verificando che $c_{38}$ distanze $6,28\sigma$ dal valore medio delle misure.

Specifichiamo che la media di queste misure, di quelle di Ramella e Redaelli, e dei due set di dati uniti sono state fatte tramite una media aritmetica. Per l'errore
abbiamo voluto tenere conto non solo della discrepanza tra le singole misure, calcolando la loro deviazione standard $\sigma$, ma anche di un eventuale errore sistematico aggiungendo
considerando il maggiore degli errori presenti sulle misure, e pertanto è stato ottenuto come somma in quadratura di questi due termini:
\begin{equation}
    \sigma c\begin{footnotesize}\textsubscript{BEST}\end{footnotesize}=\sqrt{(max(\sigma_{i}))^2+\sigma^2}
\end{equation}

Da un'analisi della distanza della misura 38 dalla Gaussiana delle misure è risultata una distanza z = 6,28 da cui abbiamo deciso di procedere a un rigetto, come mostrato
in \ref{RT} - Figura (\ref{Rigetto}).

Dopo questa serie di operazioni abbiamo ottenuto un insieme di 46 misure, ovvero quelle riportate in \ref{RT} - Figura (\ref{RT_DatiRaccolti}), con l'eccezione della 
numero 38, rigettata.

%%%%  DATI RAMELLA - REDAELLI  %%%

\subsection{Analisi dati - Misure Ramella e Redaelli}

L'apparato utilizzato da Ramella e Redaelli era invece dotato di motorino con frequenza di rotazione regolabile con continuità. Dopo avere raccolto i dati relativi alla 
lunghezza del cammino ottico abbiamo adoperato il foglio di calcolo per ricavarne un valore pari a $D = 13,60 \pm 0,01 m$ (\ref{RAM} - Figura \ref{RAM_D}).

Si precisa che durante la trattazione di questo insieme di misure verranno indicate nel testo e nelle tabelle le rotazioni dello specchio in senso orario (clockwise) come 
$CW$ e in senso antiorario (counterclockwise) come $CCW$.

La prima misura di $c$ effettuata è stata per rotazioni in senso orario.
Siccome l'equazione \ref{c_equation} è risolubile inserendo dei valori di partenza $\omega_0$ e $\delta_0$ e dei valori finali $\omega$ e $\delta$, abbiamo raccolto
i suddetti valori di partenza per poi, facendo ruotare lo specchio in senso orario con diverse frequenze, raccogliere un totale di 22 misure. Per i dati raccolti si
faccia riferimento a . Siccome abbiamo la miglior stima della velocità della luce $c_{best}$ unendo


%%%%%%%%%%%%%%%%%%%%%%%  FARE MACRO EXCEL PER INSERIRE RAPIDAMENTE SIMBOLI %%%%%%%%%%%%%%%%%%%%%%%%%%%%%%
\section{Calcolo c\begin{footnotesize}\textsubscript{BEST}\end{footnotesize}} %SIMPY

\section{Conclusioni}

\newpage

\section{Appendice}

%%%%%%%%%%%%%%%%%%%%%%%%%%%%%%%%%%%%%%%%%%%%%%%%%%%%%%%%%%%%%%%%%%%%%%%%%%%%%%%%%%%%%%%%%%%%%%%%%%%%%%%%%%%%%%

\subsection{Grafici e Tabelle - Rossi e Tambini} \label{RT}

%RT_D
\begin{figure}[h]
    \centering
    \includegraphics[width=0.5\linewidth]{RT_D.JPG}
    \caption{Misura del cammino ottico $D$}
    \label{RT_D}
\end{figure}

%RT_DatiRaccolti
\begin{figure}[h]
    \centering
    \includegraphics[width=1.0\linewidth]{RT_DatiRaccolti.JPG}
    \caption{Dati Raccolti da Rossi-Tambini. I valori di $\delta$ e $\delta_0$ misurati per una stessa frequenza di rotazione sono evidenziati col medesimo colore.}
    \label{RT_DatiRaccolti}
\end{figure}

%Tabelle_Coerenza
\begin{figure}[h]
    \centering
    \begin{subfigure}[h]{0.2\textwidth}
        \centering
        \includegraphics{Coerenza_T1.JPG}
        \caption{Misure raccolte per $\nu=750Hz$}
        \label{Tab_750}        
    \end{subfigure}
    \hfill
    \begin{subfigure}[h]{0.2\textwidth}
        \centering
        \includegraphics{Coerenza_T2.JPG}
        \caption{Misure raccolte per $\nu=1500Hz$}
        \label{Tab_1500}
    \end{subfigure}
    \hfill
    \begin{subfigure}[h]{0.2\linewidth}
        \centering
        \includegraphics{Coerenza_T3.JPG}
        \caption{Misure raccolte per $\nu=-750Hz$}
        \label{Tab_-750}        
    \end{subfigure}
    \hfill
    \begin{subfigure}[h]{0.2\linewidth}
        \centering
        \includegraphics{Coerenza_T4.JPG}
        \caption{Misure raccolte per $\nu=-1500Hz$}
        \label{Tab_-1500}
    \end{subfigure}
        \caption{Dati raccolti in tabelle secondo la $\nu$ dello specchio rotante}
        \label{Tabs}
\end{figure}

%Coerenza_Grafici
\begin{figure}[h]
    \centering
    \includegraphics[width=0.8\textwidth]{Coerenza_G1.JPG}
    \caption{Grafico relativo a Figura (\ref{Tabs}\subref{Tab_750})}
    \label{Graf_750}
\end{figure}

\begin{figure}[h]
    \centering
    \includegraphics[width=0.8\textwidth]{Coerenza_G2.JPG}
    \caption{Grafico relativo a Figura (\ref{Tabs}\subref{Tab_1500})}
    \label{Graf_1500}
\end{figure}

\begin{figure}[h]
    \centering
    \includegraphics[width=0.8\textwidth]{Coerenza_G3.JPG}
    \caption{Grafico relativo a Figura (\ref{Tabs}\subref{Tab_-750})}
    \label{Graf_-750}
\end{figure}

\begin{figure}[h]
    \centering
    \includegraphics[width=0.8\textwidth]{Coerenza_G4.JPG}
    \caption{Grafico relativo a Figura (\ref{Tabs}\subref{Tab_-1500})}
    \label{Graf_-1500}
\end{figure}

%Coerenza_Gauss
\begin{figure}
    \centering
    \begin{subfigure}{\linewidth}
        \centering
        \includegraphics[width=\linewidth]{Coerenza_Gauss1.JPG}
        \caption{Distribuzione normale relativa a Figura (\ref{Tabs}\subref{Tab_750})}
        \label{Gauss_750}
    \end{subfigure}
    \newline
    \begin{subfigure}{\linewidth}
        \centering
        \includegraphics[width=\linewidth]{Coerenza_Gauss2.JPG}
        \caption{Distribuzione normale relativa a Figura (\ref{Tabs}\subref{Tab_1500})}
        \label{Gauss_1500}  
    \end{subfigure}
    \newline
    \begin{subfigure}{\linewidth}
        \centering
        \includegraphics[width=\linewidth]{Coerenza_Gauss3.JPG}
        \caption{Distribuzione normale relativa a Figura (\ref{Tabs}\subref{Tab_-750})}
        \label{Gauss_-750}     
    \end{subfigure}
    \newline
    \begin{subfigure}{\linewidth}
        \centering
        \includegraphics[width=\linewidth]{Coerenza_Gauss4.JPG}
        \caption{Distribuzione normale relativa a Figura (\ref{Tabs}\subref{Tab_-1500})}
        \label{Gauss_-1500}   
    \end{subfigure}
    \caption{Distribuzioni normali relative a Figura (\ref{Tabs})}
    \label{Gauss}
\end{figure}

%Rigetto misura 38
\begin{figure}
    \centering
    \includegraphics[width=0.7\linewidth]{Rigetto.JPG}
    \caption{Calcolo della migliore stima della velocità della luce basata sui valori raccolti da Rossi e Tambini $c_{1,BEST}$}
    \label{Rigetto}
\end{figure}
\FloatBarrier %creo una barriera invisibile per assicurarmi che la roba sia posizionata dove la dichiaro

\subsection{Grafici e Tabelle - Ramella e Redaelli} \label{RAM}

% RAM_D
\begin{figure}[h]
    \centering
    \includegraphics[width=0.6\linewidth]{RAM_D.JPG}
    \caption{Misura del cammino ottico $D$}
    \label{RAM_D}
\end{figure}

%DATI_CW


















\end{document}


% Å ^{\circ} \vspace{1mm} per spaziare verticalmente \begin{equation} e \end{equation} con dentro un \label e fuori un \ref per numerare e avere il riferimento alle
% equazioni È


