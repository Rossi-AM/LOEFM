\documentclass{article}

\usepackage[utf8]{inputenc}
\usepackage[T1]{fontenc}
\usepackage{geometry}
\usepackage{graphicx}   %ALLOWS INSERTING IMAGES (AND MAYBE SOME MORE STUFF) %
\usepackage{caption}
\usepackage{subcaption} %ALLOWS CAPTIONS FOR SUBFIGURES%
\usepackage{verbatim} % ALLOWS MULTILINE COMMENTS WITH \begin{comment} AND \end{comment} %
\usepackage{amsmath}
\usepackage{amssymb} % PERMETTE L'USO DI SIMBOLI MATEMATICI "AVANZATI" COME 'MINORE O UGUALE' E ALTRI %
\usepackage{amstext}
\usepackage{enumitem} %PERMETTE L'USO DI ELENCHI NUMERATI
\usepackage[section]{placeins}
\usepackage{amsthm}

\geometry{a4paper}
                        %Una riga vuota tra due scritte spazia di una riga anche sul pdf. Andare a capo una volta non provoca nulla nel pdf%
\usepackage[italian]{babel}
%\usepackage[french, italian]{babel} non funzione per motivi a me ignoti
\frenchspacing 

\theoremstyle{definition}
\newtheorem{theorem}{Teorema}[section]

\theoremstyle{definition}
\newtheorem{lemma}[theorem]{Lemma}

\theoremstyle{definition}
\newtheorem{definition}{Definizione}[section]

\theoremstyle{definition}
\newtheorem{remark}{Osservazione}[section]

\theoremstyle{definition}
\newtheorem{example}{Ex}[section]



%%%  INTESTAZIONE  %%%

\title{Riassunto Analisi 2}
\author{Alessandro Matteo Rossi}
\date{\today}


%%%  INIZIO DOCUMENTO  %%%

\begin{document}

\maketitle
\tableofcontents
\newpage
%%%%%%%%%%%%%%%%%%%%%%%%%%%%%%%%%%%%%%%%%%%%%%%%%%%%%%%%%%%%
\section{Lezione 1 - 01/03/2021}


\begin{definition}[Integrale indefinito]
    Dato $\Omega \subseteq \mathbb{R}$ aperto e $f:\Omega \rightarrow \mathbb{R}$ diciamo che $f$ ammette primitiva in $\Omega$ se $\exists F:\Omega \rightarrow \mathbb{R}$ derivabile tale che $F'(x)=f(x) \; \forall x \in \Omega$. $F$ è detta primitiva di $f$.
\end{definition}

La nozione può essere estesa a $\tilde\Omega = [a,b]$ se presenti la derivata destra in $a$ e sinistra in $b$.

\vspace{3mm}

\begin{remark}
    Esistono funzioni che non ammettono primitiva. Ad esempio la funzione di Heaviside definita come $f(x)=
    \begin{cases}
    0, & \text{$x < 0$} \\
    1, & \text{$x > 0$} \\
    \end{cases}$. Infatti non esiste una funzione $F$ che abbia come derivata $f$ per ogni punto.
\end{remark}

\vspace{3mm}

In $\mathbb{R}$ parlare di \textit{aperto connesso} o \textit{intervallo} è equivalente.

\begin{remark}
    Se su $\Omega = I$ intervallo $F$ e $G$ sono primitive di $f$ su $I=(a,b)$ allora $(F-G)'=0$ $\Rightarrow$ $F-G=cost$ per Lagrange, da cui deriva la caratterizzazione delle costanti.
\end{remark}

\begin{example}[Integrale di $1/x$]
    Presa $f(x)=\frac{1}{x}$ su $\mathbb{R} \backslash \lbrace 0 \rbrace$ essa ha integrale pari a 
        $\int_{}^{} \frac{1}{x} \, dx = 
        \begin{cases}
        \log x + c & \text{$x > 0$} \\
        \log (-x) +d  & \text{$x < 0$} \\
        \end{cases}$. 
    È fondamentale non usare il valore assoluto poiche il nostro integrale è definito su intervalli, e pertanto $f$ è da integrare sui due intervalli su cui è definita.
\end{example}

Una condizione necessaria per avere primitiva è la \textbf{proprietà di Darboux} (è evidente che pertanto le funzioni derivate godano di (D)). Una funzione ha la proprietà di Darboux se mappa intervalli in intervalli.

\begin{theorem}
    Se $f$ ammette primitiva su $I$, allora $f$ gode di (D) su $I$.
    \begin{proof}[Dim]
        Siano $a,b \in I$ e sia $\gamma \in [f(a),f(b)]$ (se coincidono la tesi è ovvia!). Voglio mostrare che $\exists \, c \in [a,b]: f(c)=\gamma$, supponendo che $f(a)<\gamma < f(b)$. Sia $G(x)=F(x)-\gamma x$ dove $F'=f$. $G$ è derivabile, in quanto somma di funzioni derivabili $\Rightarrow$ è continua. La derivata di $G$ è $G'=f-\gamma$ che non è monotona ed essendo continua non è iniettiva $\Rightarrow$ non è invertibile. Allora $\exists x_1,x_2 \in (a,b): G(x_1) = G(x_2)$ e per il teorema di Rolle $\exists c \in (x_1,x_2): G'(c)=0$ e quindi $f(c) = \gamma$, che è (D).
    \end{proof}
\end{theorem}

\newpage
\section{Lezione 2 - 04/03/2021}

Esistono funzioni che pur essendo (D) non sono integrabili, a riprova del fatto che è solo una condizione necessaria.
\begin{example}
    $f(x) = \begin{cases}
        \sin \frac{1}{x} & \text{$x \neq 0$} \\
        \frac{1}{2} & \text{$x=0$} \\
    \end{cases}$
    gode di (D) ma non ammette primitiva. Se per $x=0$ valesse 0, allora avrebbe primitiva.
\end{example}

Esistono due teorie dell'integrazione: quella classica di Riemann e quella moderna di Lebesgue. La teoria moderna non permette di calcolare "più" integrali di quella classica: è migliore perchè permette di dominare l'errore più efficientemente quando si calcolano integrali approssimati (spessissimo l'integrale è solo stimabile e non calcolabile con esattezza). L'integrale classico di Cauchy-Riemann è uno strumento utile per determinare la misura di superfici o solidi.

\vspace{3mm}

Faccio tre puntualizzazioni:
\begin{enumerate}
    \item Le funzioni "buone" intese come continue, monotone e ovunque derivabili sono una assoluta minoranza. In matematica domina quella che noi consideriamo patologia. È impossibile (inteso come probabilità tendente a 0) pescare dal secchio di tutte le funzioni possibili una funzione "buona", mentre è certo (inteso come probabilità tendente a 1) pescare una funzione patologica. Appare evidente che le funzioni mai derivabili sono la quasi totalità. Se ne deduce che prendere una funzione "a caso" che sia continua e derivabile ovunque voglia dire tutt'altro che prenderne una "a caso". 
    
    Riusciamo però a dominare la matematica con le poche funzioni "buone" rimaste, che sono continue e derivabili, perchè posso approssimare bene quanto voglio una funzione patologica con una buona (con deboli ipotesi), tanto quanto posso approssimare un trascendente (probabilità 1 se pesco tra i numeri) con un razionale (probabilità 0).
    \item Denoto con $Y^{X}$ l'insieme di tutte le funzioni $f:X\rightarrow Y$.
    \item D'ora in poi, se non diversamente specificato, considererò intervalli chiusi e limitati $[a,b]\subset \mathbb{R}$ e funzioni $f$ limitate.
\end{enumerate}

\begin{definition}[Partizione]
    Una partizione di $[a,b]$ è un insieme ordinato di $n+1$ punti casuali. $P=\lbrace x_0,x_1, ...,x_n \rbrace$ t.c. $a=x_0 < x_1 < x_2 <...<x_n=b$.

    Il massimo delle ampiezze degli intervalli $\max\lbrace\Delta x_i\rbrace$ è detto \textbf{taglia della partizione}, dove $\Delta x= x_i - x_{i-1}$ con $i=1,... ,n$.
\end{definition}

Per arrivare a parlare di integrale inteso come area sottesa al grafico di una funzione è necessario introdurre i concetti di somme superiori e inferiori.

\vspace{5mm}

Considero una funzione $f:[a,b]\rightarrow \mathbb{R}$ limitata sull'intervallo $I=[a,b]$. Sia $P=\lbrace x_0,...,x_n \rbrace$ una partizione di $[a,b]$. Scriviamo $M_i=\underset{x\in [x_{i-1},x_i]}{\sup} f(x)$ e $m_i =\underset{x\in [x_{i-1},x_i]}{\inf} f(x)$ e definiamo le \textbf{somme superiori} come $S(P,f)=\displaystyle\sum_{i=0}^{n}M_i \Delta x_i$ e le \textbf{somme inferiori} come $s(P,f)=\displaystyle\sum_{i=0}^{n}m_i \Delta x_i$, relative alla partizione $P$.

\begin{remark}[$P$ e $\mathcal{P}$]
    Considerando $\mathcal{P}$, l'isnieme di tutte le partizioni $P$ di $I=[a,b]$ ho che 
    \[
    M(b-a) \geq S(P,f) \geq s(P,f) \geq m(b-a) \;\;\; \forall P \in \mathcal{P}
    \]
\end{remark}

Da qui chiamo \textbf{integrale superiore} e \textbf{integrale inferiore} le scritture
\[
    \overline{\int_{a}^{b}} f(x) dx= \underset{P\in\mathcal{P}}{\inf} S(f,P) \;\;\;\;\;\;\;\;\; \underline{\int_{a}^{b}} f(x) dx= \underset{P\in\mathcal{P}}{\sup} s(f,P)
\]

È evidente che $\overline\int_{a}^{b} f(x) dx \geq \underline\int_{a}^{b} f(x) dx$

\begin{example}[Funzione di Dirichlet]
    $f(x) = \begin{cases}
        1 & x\in[a,b]\cap \mathbb{Q} \\
        0 & x\in[a,b]\; \backslash\; \mathbb{Q} \\
    \end{cases}$
    In questo caso la disuguaglianza tra integrale superiore e inferiore è stretta, poichè $\underset{I}{\max}f=1$ e $\underset{I}{\min}f=0$.
\end{example}

\begin{definition}[Integrale di Riemann]
    Sia $f:[a,b]\rightarrow \mathbb{R}$ limitata. Diciamo che $f$ è \textbf{R-integrabile} o $f\in \mathcal{R}([a,b])$ se l'integrale superiore coincide con l'integrale inferiore.
    \[
        \overline{\int_{a}^{b}} f(x) dx = \underline{\int_{a}^{b}} f(x) dx \doteq \int_{a}^{b} f(x)dx
    \]
    \label{Riemann}
\end{definition}

Si può semplificare questa definizione con una caratterizzazione delle funzioni R-integrabili che renda la definizione più facile. Prima però un risultato preliminare.

\begin{lemma}[Raffinamento di una partizione]
    Una partizione $P^*=P \bigcup \lbrace\xi_1,...,\xi_n \rbrace$ che si ottiene aggiungendo un numero finito di punti a $P$ si dice \textbf{raffinamento}.
    
    Se $P^*$ è un raffinamento di $P$, allora $S(P,f) \geq S(P^*,f) \geq s(P^*,f) \geq s(P,f)$.
    \begin{proof}[Dim]
        Per dimostrarlo è sufficiente aggiungere un solo punto alla partizione. Se $\xi \in (x_{i-1},x_i)$ allora $\underset{x \in [x_{i-1},x_i]}{\sup} f(x) = \max \lbrace \underset{x \in [x_{i-1},\xi]}{\sup} f(x), \underset{x \in [\xi,x_i]}{\sup} f(x) \rbrace$

        \vspace{3mm}

        quindi

        \vspace{2mm}

        $M_i \Delta x_i = \underset{x \in [x_{i-1},x_i]}{\sup} f(x) \Delta x_i = \underset{x \in [x_{i-1},x_i]}{\sup} f(x)(x_i -\xi + \xi - x_{i-1}) \geq \underset{x \in [x_{i-1},\xi]}{\sup} f(x)(\xi-x_{i-1}) + \underset{x \in [\xi,x_i]}{\sup} f(x)(x_i- \xi)$.

        È quindi evidente che aggiungendo punti alla partizione le somme superiori descrescano e quelle inferiori crescano.
    \end{proof}
\end{lemma}

\begin{theorem}[Criterio per la R-integrabilità]
    Sia $f:[a,b]\rightarrow \mathbb{R}$ limitata. Allora $f\in\mathcal{R}([a,b]) \Leftrightarrow \forall\varepsilon >0 \; \exists P\in\mathcal{P}: S(P,f)-s(P,f)<\varepsilon$.
    \begin{proof}[Dim]

        ($\Leftarrow$)

        Per ogni partizione $P$ abbiamo $s(P,f) \leq \underline{\int_{}^{}}f \leq \overline{\int_{}^{}}f \leq S(P,f)$ da cui $S(P,f)-s(P,f) \geq \overline{\int_{}^{}}f-\underline{\int_{}^{}}f$.

        Da cui, per ipotesi,
        $\forall \varepsilon > 0 \; \exists P: \varepsilon > S(P,f) - s(P,f) \geq \overline{\int_{}^{}}f-\underline{\int_{}^{}}f$. Al limite integrale inferiore e superiore coincidono soddisfando Def. \ref{Riemann}.

        ($\Rightarrow$) Se $f$ è R-integrabile allora $\int_{a}^{b}f = \overline{\int_{}^{}} f = \underline{\int_{}^{}} f$. Poichè l'integrale superiore è $\inf$ delle somme superiori, per ogni $\varepsilon > 0 \overline{\int_{}^{}}f+ \varepsilon/2$ non è minorante: esiste una partizione $P_1$ tale che $S(P_1,f)<\overline{\int_{}^{}}f+ \varepsilon/2$. Ragionamento analogo, con i dovuti cambi di segno, si può fare per le somme inferiori e con una partizione $P_2$. Pertanto prendendo $P = P_1 \bigcup P_2$ raffinamento di queste partizioni per il Lemma si ha la descrescita delle somme superiori e la crescita delle somme inferiori e quindi la tesi.
    \end{proof}
\end{theorem}


\newpage
\section{Lezione 3 - 10/03/2021}

Vediamo una serie di condizioni sufficienti affinchè una funzione limitata sia R-integrabile su un intervallo $I$ chiuso e limitato.

\begin{theorem}[Condizioni sufficienti per l'integrabilità]
    Sia $f:[a,b]\rightarrow \mathbb{R}$ una funzione limitata. Allora
    \begin{enumerate}
        \item $f$ è continua $\Rightarrow f \in \mathcal{R}([a,b])$
        \item $f$ è monotona $\Rightarrow f \in \mathcal{R}([a,b])$
        \item $f$ ha un numero finito di punti di discontinuità $\Rightarrow f \in \mathcal{R}([a,b])$
    \end{enumerate}
    
    \begin{proof}[Dim. 1]
        Se $f$ è continua su $[a,b]$ allora per Heine-Cantor è uniformemente continua, cioè
        \[
        \forall \varepsilon > 0 \; \exists \delta>0: \forall x,y: \left|x-y \right| < \delta, \left|f(x)-f(y)\right| < \frac{\varepsilon}{b-a}
        \]
        Presa ora una partizione $P=\lbrace x_0,...,x_n \rbrace$ di $[a,b]$ di taglia $\Delta x_i < \delta$. Poichè $f$ è continua su ogni subintervallino $[x_{i-1}, x_i]$ esistono $s_i$ e $t_i$ tali che $M_i=f(s_i)$ e $m_i=f(t_i)$, cioè sup e inf sono assunti per Weierstrass. 

        Ma allora la differenza tra somme superiori e inferiori è

        \[
        S(P,f) - s(P,f) = \displaystyle\sum_{i=0}^{n} (M_i-m_i)\Delta x_i < \frac{\varepsilon}{b-a}\displaystyle\sum_{i=0}^{n} \Delta x_i = \varepsilon
        \]  

        il che è equivalente al criterio di R-integrabilità, da cui si deduce che la funzione sia per l'appunto R-integrabile.
    \end{proof}
    \begin{proof}[Dim. 2]
        Prendo $f$ monotona crescente (ad esempio) e una partizione $P=\lbrace x_0,...,x_n \rbrace$ dell'intervallo $[a,b]$ tale che la sua taglia sia $\Delta x < \frac{\varepsilon}{f(b)-f(a)}$.

        Poichè $f$ è monotona il massimo e il minimo sulla partizione sono $M_i=f(x_i)$ e $m_i=f(x_{i-1})$. Allora la differenza fra somme superiori e inferiori è \[S(P,f) - s(P,f) = \sum_{i=0}^{n}(M_i-m_i)\Delta x_i < \frac{\varepsilon}{f(b)-f(a)} \sum_{i=0}^{n}[f(x_i)-f(x_{i-1})] = \varepsilon\]

        Che altro non è che la condizione di integrabilità.
    \end{proof}
    \begin{proof}[Dim. 3]
        In questo caso è fondamentale lavorare con una funzione limitata. Dimostro per un punto e poi iterando posso dimostrarlo per un insieme finito di punti. Suppongo di avere un'unica discontinuità in $c \in [a,b]$. Fissato $\varepsilon>0$ considero due punti $c-\delta$ e $c+\delta$ con $\delta$ infinitesimo. La funzione è quindi continua sugli intervallini $[c-\delta,c]$ e $[c,c+\delta]$.
    \end{proof}
\end{theorem}

\begin{remark}
    I punti (1) e (3) sono molto simili. Infatti essere continua vuol dire avere un numero finito di punti di discontinuità, cioè 0.
\end{remark}

\begin{remark}
    Nel caso di funzione continua (1) non è necessario specificare la limitatezza della funzione, poichè per Weierstrass l'immagine di un compatto [a,b] è compatta, e quindi provvista di massimo e minimo.
\end{remark}

\begin{remark}
    Lavorare su intervalli chiusi non è necessario finchè la funzione è limitata. L'intervallo compatto serve per garantire l'esistenza di estremi finiti per Weierstrass.
\end{remark}
\newpage
\section{Tavola degli integrali}

Gli integrali notevoli sono ottenibili leggendo la tabella delle derivate al contrario.
\begin{center}
    \begin{tabular}{||c|c|c||}
        \hline
        $f$ & $\int_{}^{} f$ & C.E. \\
        \hline\hline
        0 & $c \in \mathbb{R}$ & \\
        $x^n$ & $\frac{x^{n+1}}{n+1} + c $ & $ n \in \mathbb{N}_0, x \in \mathbb{R}$ \\
        $x^\alpha$ & $\frac{x^{\alpha+1}}{\alpha+1}+c$ & $ \alpha \in -1, x > 0, \alpha \in \mathbb{R}$ \\
        $\frac{1}{x}$ & $\begin{cases}
                        \log x + c & \text{$x > 0$} \\
                        \log (-x), & \text{$x < 0$} \\
                        \end{cases}$ & \\
        $e^{x}$ & $e^{x} + c$ & $x \in \mathbb{R}$ \\
        $\sin x$ & $-\cos x + c$ & $x \in \mathbb{R}$ \\
        $\cos x$ & $\sin x + c$ & $x \in \mathbb{R}$ \\
        $Ch x$ & $Sh x + c$ & $x \in \mathbb{R}$ \\
        $Sh x$ & $Ch x + c$ & $x \in \mathbb{R}$ \\
        $1 + \tan^2 x = \frac{1}{\cos^2 x}$ & $\tan x + c$ & $x \in (-\frac{\pi}{2}, \frac{\pi}{2})$ \\
        $\frac{1}{1+x^2}$ & $\arctan x + c$ & $x \in \mathbb{R}$ \\
        $\frac{1}{\sqrt{1-x^2}}$ & $\arcsin x + c$ & $x \in (-1,1)$ \\
        $-\frac{1}{\sqrt{1-x^2}}$ & $\arccos x + c$ & $x \in (-1,1)$ \\
        \hline
    \end{tabular}
\end{center}

%È
\end{document}


